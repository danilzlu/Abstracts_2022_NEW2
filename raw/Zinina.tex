\documentclass[12pt]{article}

\title{\bf\Large Morse-Bott energy function for topological flows with a hyperbolic chain-recurrent set consisting of a finite number of orbits}
\author{Zinina S. Kh.}
\date{}


\begin{document}
	
	\begin{center}
		\maketitle
		{\large\textit{Ogarev Mordovia State University}}
	\end{center}
	
	
\smallskip

Regular topological dynamical systems are defined as dynamical systems whose chain-recurrent set is topologically hyperbolic and consists of a finite number of fixed points and periodic orbits. For such systems,  provides an exhaustive description of the behavior of invariant manifolds of chain components, both from the point of view of asymptotics and from the point of view of the topology of their embedding in the carrier manifold.

Also it is proved that for a regular flow without periodic orbits, given on a topological manifold of any dimension, there exists a (continuous) Morse energy function. The result obtained is an ideological continuation of the work of S. Smale \cite{1}, in which he established the existence of a smooth energy Morse function for any gradient-like flow on a manifold, and a partial solution of the Morse problem on the existence of continuous Morse functions on any topological manifolds. Namely, a topological manifold admits a continuous Morse function if and only if it admits a regular topological flow without periodic orbits. This result was obtained in the present work within the framework of constructing a continuous Morse-Bott energy function for an arbitrary continuous regular flow on a topological manifold, and is an analogue of the theorem
K. Meyer \cite{2}, who in 1968 constructed the Morse-Bott energy function for an arbitrary Morse-Smale flow on a smooth closed $n$-manifold.

We proves the existence of a continuous energy function for any regular flow. This is result are the ideological continuation of the works of S. Smale \cite{1} and K. Meyer \cite{2} on the existence of the Morse energy function for gradient-like flows and the Morse-Bott energy function for Morse-Smale flows, respectively.

	
A function $\varphi$ is called a {\it continuous Morse-Bott function} if any connected component of the set $Cr_\varphi$ is either a non-degenerate critical point or belongs to a non-degenerate critical submanifold.
	
	{\bf Statement}.
	{\it Any regular topological flow $f^t:M^n\to M^n$ without periodic orbits has a continuous energy Morse function}.
	
	The concept of a continuous Morse function was introduced by Morse back in 1959 in \cite{5}, at the same time the validity of the Morse inequalities was proved for it, and later (in \cite{6}) a number of properties similar to the properties of the smooth Morse function . However, the question of the existence of a continuous Morse function on an arbitrary topological manifold is still an open question. Since the continuous Morse function generates a topological gradient-like flow on the manifold \cite{6}, then 	Statement is a partial solution of the Morse problem: a topological manifold admits a continuous Morse function if and only if it admits a topological flow with a finite hyperbolic chain-recurrent set.
	
	Statement  follows directly from a more general result.
	
	{\bf Theorem}.
	{\it Any regular flow $f^t\in G^t$ has a continuous energy Morse-Bott function whose critical points are either non-degenerate or form non-degenerate one-dimensional manifolds.}
	
	The existence of an energy function fundamentally distinguishes continuous regular systems from discrete ones. For the latter, an obstacle to the construction of the energy function is the possible presence of wild saddle separatrices discovered by D. Pixton \cite{7} in 1977 in dimension three. Examples of regular flows with wild separatrices are also known; such flows are constructed, for example, in recent works by V. Medvedev and E. Zhuzhoma \cite{8}. However, it follows from the results that for regular flows, the wildness of the separatrices is not an obstacle to the existence of the Morse energy function.



{\it Thanks.} This work is supported by the grant RFBR, project number 20-31-90069 and by Foundation for the Advancement of Theoretical Physics and
Mathematics BASIS (project 19-7-1-15-1).

\begin{thebibliography}{99}

\bibitem{1} S. Smale On gradient dynamical systems // Ann. of Math. (2). 1961. - Vol. 74. - P.~199--206.

\bibitem{2}  K. R. Meyer Energy functions for Morse Smale systems // Amer. J. Math. - 1968. - Vol. 90. - P. 1031--1040.

%\bibitem{3} O. V. Pochinka, S. Kh.  Zinina  Construction of the Morse�Bott Energy Function for Regular Topological Flows // Regular and Chaotic Dynamics. - 2021. - Vol. 26, No.~4.~- P. 350--369.

\bibitem{4} T. V. Medvedev, O. V. Pochinka, S. Kh. Zinina  On existence of Morse energy function for topological flows // Advances in Mathematics. - 2021. - Vol. 378. - 107518.

\bibitem{5} M. Morse  Topologically non-degenerate functions on a compact n-manifold // J. Analyse Math. - 1959. - Vol. 7. - P. 189--208.

\bibitem{6} R. C. Kirby, L. C. Siebenmann  Foundational Essays on Topological Manifolds, Smoothings, and Triangulations, (AM-88), Vol. 88. - Princeton University Press.~- 2016.

\bibitem{7}  D. Pixton  Wild unstable manifolds // Topology. - 1977. - Vol. 16., No. 2. - P. 167--172.

\bibitem{8}  V. S. Medvedev, E. V. Zhuzhoma   Morse-Smale systems with few nonwandering points // Topology Appl. - 2013. -  Vol. 3 (160). - P. 498--507.

\end{thebibliography}
\end{document}



