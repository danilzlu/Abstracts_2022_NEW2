\documentclass[12pt]{article}

\title{\bf\Large Extreme States of Interacting Solitons}
\author{\underline{Slunyaev A.V.}, Tarasova T.V., Didenkulova E.G., Pelinovsky E.N.}
\date{}


\begin{document}
	
\begin{center}
	\maketitle
	{\large\textit{National Research University Higher School of Economics, N. Novgorod, Russian Federation}}
	
	{\large\textit{Institute of Applied Physics of the Russian Academy of Sciences, N. Novgorod, Russian Federation}}
\end{center}

Interacting solitons (soliton gas or soliton turbulence) represent the opposite limit of interacting linear waves which possess the Gaussian statistics. It is in the focus of current research related to the problem of extreme phenomena (including the so-called rogue waves) in irregular waves, when the fraction of coherent states cannot be neglected. Due to the inherent nonlinearity, the theoretical description of a soliton gas in the general formulation is difficult and is still an open problem \cite{PelinovskyShurgalina2017}. The kinetic description of ensembles of solitons cannot provide the requested information about the wave amplitudes.

Simplified model problems are considered as a constructive approach to solving the problem. In particular, in a rarefied soliton gas, it is usually sufficient to take into account only pairwise soliton interactions. Collisions between pairs of solitons were understood in \cite{Pelinovskyetal2013} as "elementary acts" of soliton turbulence. Indeed, the picture of pair interactions agrees qualitatively with the results of direct numerical simulations of large ensembles of solitons.
%
At the same time, multiple soliton collisions were also observed in numerical simulations \cite{GelashAgafontsev2018,Didenculova2019}. Due to the strong nonlinearity, these situations are difficult for the direct numerical simulation \cite{Didenkulovaetal2019}. However, they can lead to the generation of extremely high waves when solitons have different signs \cite{SlunyaevPelinovsky2016,Slunyaev2019}.

Solitons of the same sign (as in the classical Kortweg--de Vries equation) do not form larger waves during interaction, but can form states with a critical (maximum) density \cite{PelinovskyShurgalina2017,El2016}. In this paper, we discuss some recent own results on the study of soliton states with a critical density. The exact solutions of the Korteweg--de Vries equation constructed using the original numerical subroutine are investigated. Surprisingly, some key characteristics of such soliton fields can be calculated analytically. These results should help further understanding of extreme soliton states and, eventually, the development of an appropriate theoretical description.

The research was partly funded by Laboratory of Dynamical Systems and Applications NRU HSE, of the Ministry of science and higher education of the RF grant ag. No 075-15-2019-1931 and by RSF grant 19-12-00253.

\begin{thebibliography}{99}

\bibitem{PelinovskyShurgalina2017}
E. Pelinovsky, E. Shurgalina. KDV soliton gas: interactions and turbulence. In: Challenges in Complexity: Dynamics, Patterns, Cognition (Eds.: I. Aronson, A. Pikovsky, N. Rulkov, L. Tsimring), Springer, Vol. 20, 295--306 (2017).

\bibitem{Pelinovskyetal2013} 
E.N. Pelinovsky, E.G. Shurgalina, A.V. Sergeeva, T.G. Talipova,
G.A. El, R.H.J. Grimshaw.
Two-soliton interaction as an elementary act of soliton turbulence
in integrable systems.
Phys. Lett. A 377, 272--275 (2013).

\bibitem{GelashAgafontsev2018}
A.A. Gelash, D.S. Agafontsev. Strongly interacting soliton gas and formation of rogue
waves. Phys. Rev. E 98, 042210 (2018).

\bibitem{Didenculova2019} 
E.G. Didenkulova. Numerical modeling of soliton turbulence within the focusing Gardner equation: Rogue wave emergence. Physica D 399, 35--41 (2019).

\bibitem{Didenkulovaetal2019}
E.G. Didenkulova, A.V. Kokorina, A.V. Slunyaev. Numerical simulation of soliton gas within the Korteweg -- de Vries type equations. Comput. Technol. 24, 52--66 (2019).

\bibitem{SlunyaevPelinovsky2016}
A.V. Slunyaev, E.N. Pelinovsky. The role of multiple soliton interactions in generation of rogue waves: the mKdV framework. Phys. Rev. Lett. 117, 214501 (2016).

\bibitem{Slunyaev2019}
A. Slunyaev. On the optimal focusing of solitons and breathers in long wave models. Stud. Appl. Math. 142, 385--413 (2019).

\bibitem{El2016}
G.A. El. Critical density of a soliton gas. Chaos 26, 023105 (2016).
	
\end{thebibliography}

\end{document}
