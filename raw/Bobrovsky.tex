\documentclass[12pt]{article}
\usepackage{authblk}

\title{\bf\Large Shilnikov attractors in a minimal network of coupled neuron models}
\author{Bobrovsky A.A., Shchegoleva N.A., \underline{Stankevich N.V.}}
\date{}



\begin{document}
	
\begin{center}
	\maketitle
	{\large\textit{HSE University - Nizhny Novgorod, Russia}}
\end{center}

In the work we consider minimal ensemble of Hodgkin-Huxley-type of neuron models. Typical for such systems is that the anti-phase synchronization is suspected here. In works [1-3], it was shown that, taking into account various types of coupling between subsystems (both excitatory and inhibitory), it is possible to single out in-phase synchronization. It is also shown that multistability between in-phase and anti-phase oscillatory modes is possible in this system. Also such models can demonstrate in-phase and anti-phase synchronous bursting behavior. Both bursting and spiking attractors can be chaotic. We analyze chaotic dynamics and reveal different types of chaos including hyperchaos. We study in detail route to hyperchaos which associated with the torus destruction and formation Shilnikov attractors [4]. 

As a base for analysis we use Hodgkin-Huxley-type models: Hindmarsh-Rose model and Sherman model. We consider different parameters of sub-systems, when they demonstrate bursting oscillations of different type, or spiking oscillations. We analyse parameter space for different type of coupling (inhibitory and excitatory). We have shown the formation of Shilnikov chaotic attractors which leads to the appearance hyperchaos [5].

The work is supported by the grant of Russian Science Foundation (Project No. 20-71-10048).

\begin{thebibliography}{99}
    \bibitem{st1} Izhikevich EM. Dynamical systems in neuroscience. MIT press; 2007.
    \bibitem{st3} Jalil S., Belykh I., Shilnikov A. Physical Review E. 2012. Vol. 85. No. 3. PP. 036214.
	\bibitem{st5} Belykh I., Reimbayev R., Zhao K. Physical Review E. 2015. Vol. 91. No. 6. PP. 062919.
	\bibitem{st6} Gonchenko A., Gonchenko S., Kazakov A., Turaev D. (2014) // Int. J. of Bif. and Chaos, 24(08), 1440005.
	\bibitem{st7} Stankevich N., Kazakov A., Gonchenko S. (2020) // Chaos, 30(12), 123129.
	
\end{thebibliography}

\end{document}
 

\end{document}