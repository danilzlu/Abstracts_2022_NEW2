
	
\begin{center}
	\maketitle
	{\large\textit{Crimean Federal University,\\
			Simferopol'}}
\end{center}

In this report, we will try to give an overview of the current state of the theory of bifurcations of solutions to impulsive systems.

Many evolution processes are subject to short-term perturbations whose duration is negligible in
comparison with the duration of the process. Impulsive differential equations~\cite{samoilenko-perestyuk1987, samoilenko-perestyuk1995} are powerful tools to model this type of evolution processes. Impulsive differential equations see applications in numerous fields where the systems of study exhibit rapid jumps in state. Such jumps may be intrinsic to the system,
such as in the firing of a neuron in a biological neural network, or synthetic, such as the application of
an insecticide or antibiotic treatment in a biological model. One of the most common applications of
the theory of impulsive differential equations arises in the case, where a continuous autonomous
system is perturbed by impulses in an impulsive control setting~\cite{Lakmeche_Arino2001, Xie2017}. Despite the large number of publications devoted to impulsive systems of the considered type~\cite{Hristova}-\cite{Anashkin2021}, the qualitative theory, in particular, the theory of bifurcations  is far from being sufficiently developed. Only recently there have appeared very meaningful studies devoted to this direction~\cite{Church_Liu2017}-\cite{Akhmet_Kashkynbayev2017}.

Mathematical models are considered, the results of computer modeling and examples of applications are given.

\begin{thebibliography}{99}
\bibitem{samoilenko-perestyuk1987}
A. M. Samoilenko and M. O. Perestyuk,
Differentsial'nye uravneniya s impul'snym vozdeistviem, Kyev: Vysha Shkola (1987) (in Russian).
\bibitem{samoilenko-perestyuk1995}
A. M. Samoilenko and M. O. Perestyuk,
Impulsive differential equations, River Edge: World Scientific (1995).
\bibitem{Lakmeche_Arino2001}
A. Lakmeche and O. Arino, Nonlinear mathematical model of pulsed-therapy of heterogeneous
tumors, Nonlinear Anal. Real World Appl., {2}, 455-465 (2001).
\bibitem{Xie2017}
Y. Xie, L. Wang, Q. Deng, Z. Wu,  The dynamics of an impulsive predator-prey model with communicable disease in the prey species only, Appl. Math. Comput. 292, 320-335 (2017).
\bibitem{Hristova}
S. G. Hristova, 
Qualitative investigations and approximate methods for impulsive equations. New York: Nova Science Publishers, Inc. (2009).
\bibitem{Akhmet2010}
M. Akhmet, Principles of Discontinuous Dynamical Systems. New-York: Springer, (2010).
\bibitem{Hu_Han2009}
Z. Hu, M. Han.
Periodic solutions and bifurcations of first-order periodic impulsive differential equations.
International Journal of Bifurcation and Chaos 19, No.8, 2515-2530 (2009).
\bibitem{Anashkin2021}
O.V. Anashkin and O.V. Yusupova, Stability in the Critical Case and Bifurcations in Impulsive Systems. Lobachevskii J Math 42, 3574-3583 (2021). https://doi.org/10.1134/S1995080222030039
\bibitem{Church_Liu2017}
K.E.M. Church and Xinzhi Liu.
Bifurcation Analysis and Application for Impulsive Systems with Delayed Impulses, International Journal of Bifurcation and Chaos {27},  1750186 (2017).
\bibitem{Church_Liu2021}
K.E.M. Church and X. Liu. Bifurcation theory of impulsive dynamical systems. IFSR
International Series in Systems Science and Systems Engineering, Vol. 34. Springer
Nature (2021).
\bibitem{Akhmet_Kashkynbayev2017}
M. Akhmet, A. Kashkynbayev,
Bifurcation in Autonomous and Nonautonomous
Differential Equations with Discontinuities.
Springer Nature Singapore Pte Ltd. and Higher Education Press (2017).

\end{thebibliography}

