

\begin{center}
	\maketitle
	{\large\textit{NRU HSE}}
\end{center}

%Ultrasound contrast agents are simulated microbubbles
We study dynamics in the model describing oscilaltions of three coupled encapsulated microbubbles under the influence of external pressure field ~\cite{Takahira1995, Doinikov2004}. We study three configuration of bubbles: in the vertices of an equilateral triangle, in the vertices of an isosceles triangle and in the nodes of a chain with equal distances between the neighboring bubbles.

In the symmetrical case of equilateral triangle, the dynamics can be fully synchronous, partially synchronous, or asynchronous. In the other two spatial configurations it could be either partially synchronous or asynchronous. We find that the main mechanism of destruction of synchronization in the completely symmetrical configuration is the bubbling transition scenario. An interesting feature of the implementation of this scenario in this model is that the asynchronous component of the trajectory has three positive Lyapunov exponents.
On the other hand, the partial synchronization in case of the chain or an isosceles triangle configurations, is usually destroyed via a special case of period-doubling bifurcation, when the initial limit cycle becomes unstable in the direction transversal to the partial synchronization hyperplane.

We show that fully synchronous or partially synchronous chaotic attractors usually emerge via a period-doubling cascade. While asynchronous chaotic regimes emerge either via a period--doubling cascade of an asynchronous limit cycle or the Afraimovich--Shilnikov scenario, starting from a partially synchronous limit cycle. Hyperchaotic attractors with two positive Lyapunov exponents typically occur via the scenario involving emergence of the discrete Shilnikov attractor ~\cite{Garashchuk2019}. We also observed hyperchaotic attractors with three positive Lyapunov exponents in different spatial configurations, although the details of scenario of their emergence are not yet clear.


%~\cite{Rian}


\begin{thebibliography}{99}
\bibitem{Takahira1995} H. Takahira, S. Yamane, and T. Akamatsu, JSME Int. J. Ser. B 38, 432 (1995).
\bibitem{Doinikov2004} N.A. Pelekasis, A. Gaki, A. Doinikov, and J.A. Tsamopoulos, J. Fluid Mech. 500, 313 (2004).
\bibitem{Garashchuk2019} I.R. Garashchuk, D.I. Sinelshchikov, A.O. Kazakov, and N.A. Kudryashov, Chaos An Interdiscip. J. Nonlinear Sci. 29, 063131 (2019).
\end{thebibliography}

