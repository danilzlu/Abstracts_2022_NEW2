
	
\begin{center}
	\maketitle
	{\large\textit{1. Financial University under the Government of the Russian Federation}}
	
	{\large\textit{2. Moscow Institute of Physics and Technology (National Research University)}}
	
	{\large\textit{3. Institute of Machines Science, Russian Academy of Sciences}}
\end{center}


The present report is review research of the bifurcation analysis of the mechanical model of a Lagrange top with a vibrating suspension point.
Mechanical system  describes the dynamics of a rigid body in a uniform gravity field. One of the points of the body lying on the axis of symmetry (the suspension point) performs high-frequency vertical oscillations of small amplitude.  System of differential equations is a completely Liouville-integrable Hamiltonian system with two degrees of freedom. Such a system can be subjected to bifurcation analysis and clearly demonstrate the problems of stability research based on the analysis of the type of singularities. The type of rank-zero singularities of the integral mapping, which are associated with equilibrium positions, are determined in the paper \cite{BorRyabSok2020}. In contrast to the classical approach used for stability analysis of the upper equilibrium in \cite{Markeev2012}, an analysis of the type of singularities of the integral mapping revealed relations under which the lower equilibrium position becomes unstable. Additionally, a unique phenomenon is observed in the considered mechanical system, namely, the realization of doubly pinched torus. The bifurcation diagram and the atlas of bifurcation diagrams are explicitly defined. The problem of stability of regular precessions is carried out on the basis of determining the type of singularity and geometric interpretation of stability on the bifurcation diagram. The regular precessions will be unstable for the branch of the bifurcation curve between the cusped points. At the remaining points of the bifurcation curve, the regular precessions have an elliptical type, which corresponds to the stability of regular precessions.

The research was supported by the Russian Science Foundation, grant no. 19-71-30012 and by Russian Foundation for Basic Research, grant no. 20-01-00399A.


\begin{thebibliography}{99}
	\bibitem{BorRyabSok2020} A.~V.~Borisov,  P.~E.~Ryabov, and S.~V.~Sokolov,
``On the Existence of Focus Singularities in One Model of a Lagrange Top with a Vibrating Suspension Point'', Doklady Mathematics, vol.~102, no.~3. pp.~468--471.

\bibitem{Markeev2012} A.~P.~Markeev, ``On the motion of a heavy dynamically symmetric rigid body with vibrating suspension point'', Mechanics of Solids, 2012, vol.~47, no.~4. pp.~373--379.
\end{thebibliography}

